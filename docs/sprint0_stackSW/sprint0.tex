\documentclass[compress]{beamer}
\usepackage[
    title={Stack Sotware},
    subtitle={Sprint 0},
    event={Sprint 0},
    author={Daniele La Prova},
    longauthor={Daniele La Prova 0320429},
    email={daniele.laprova@students.uniroma2.eu},
    institute={CSW 2022-2023},
    longinstitute={Universita' degli Studi di Roma Tor Vergata},
]{unislides}
\usepackage{hyperref}
\usepackage{amssymb}
\usepackage{minted}

\begin{document}
    
\begin{frame}[plain]
    \titlepage
\end{frame}

\section{Github}
\subsection{Organizzazione}
\begin{frame}{\subsecname}
    \begin{itemize}
        \item La nostra organizzazione si chiama \href{https://github.com/CSW-Teams}{CSW-Teams}
        sotto alla quale è registrato il nostro team \href{https://github.com/orgs/CSW-Teams/teams/sprintfloyd}{SprintFloyd};
        \item È stata creata una repository per \href{https://github.com/CSW-Teams/MS3}{MS3}
        a cui sono stati assegnati tutti i membri di SprintFloyd;
        \item Nella cartella \texttt{docs} sono presenti i sorgenti delle documentazioni per uffaPoints e queste slides
        (decidiamo se mettere tutto qua o altrove);
    \end{itemize}
\end{frame}
\begin{frame}
    \frametitle{\subsecname}
    \begin{itemize}
        \item In futuro, altri frequentanti al corso potranno essere invitati
        all'organizzazione con pieni poteri amministrativi (Owner) e creare così un loro
        team;
        \begin{itemize}
            \item[\checkmark] Ciò permette a ogni team di personalizzarsi come vuole e sviluppare così cameratismo;
            \item[\checkmark] È molto facile assegnare le persone di un team a una repository;   
        \end{itemize}
        \item È necessario che un membro del corso che non sia un alunno sia iscritto all'organizzazione in modo
        che possa aggiungere alunni futuri all'organizzazione e creare così un loro team.
    \end{itemize}
\end{frame}

\section{Spring Boot}
\begin{frame}{\secname}
    \begin{itemize}
        \item È possibile usare \texttt{application.properties} per configurare il
        progetto con valori base delle proprietà, e farne l'override o aggiungerne di nuove
        specificando altri profili come ad esempio \texttt{application-debug.properties};
        \item È necessario specificare valori per entità sensibili attraverso le variabili
        d'ambiente.
    \end{itemize}
\end{frame}
\begin{frame}[fragile]
    \frametitle{\secname}
    \begin{minted}[autogobble, fontsize=\scriptsize, linenos, numbersep=3pt]{json}
        {
            "version": "0.2.0",
            "configurations": [
                {
                    "type": "java",
                    "name": "Debug Ms3Application",
                    "request": "launch",
                    "mainClass": "org.cswteams.ms3.Ms3Application",
                    "projectName": "ms3",
                    "env": {
                        "SPRING_PROFILES_ACTIVE": "debug",
                        "MYSQL_HOST": "localhost",
                        "MYSQL_SCHEMA": "hibernate_db",
                        "SPRING_DATASOURCE_USERNAME": "testhibernatespring",
                        "SPRING_DATASOURCE_PASSWORD": "mypassword",
                    }
                }
            ]
        }
    \end{minted}  
\end{frame}

\section{React}

\begin{frame}
    \frametitle{\secname}

    %TODO: Document minimal working example with react

\end{frame}

\end{document}